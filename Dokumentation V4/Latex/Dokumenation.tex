\documentclass[
    DIV12,
    cleardouble=plain,
    headings=normal,
    pdftex,
    headexclude,footexclude,
    final
]{scrreprt}


%\usepackage{spreadtab}
\usepackage{xspace}
\usepackage[ngerman]{babel}
\usepackage[utf8]{inputenc}
%\usepackage[T1]{fontenc}
\usepackage[pdftex]{graphicx}
\usepackage[bookmarks]{hyperref}
\usepackage{scrpage2}
\usepackage{longtable}
\usepackage{caption}
\usepackage{pgfplots}
\usepackage{float}
\usepackage{xcolor}
\usepackage{colortbl}
\usepackage{tabularx}
\usepackage{multirow} % tabelle
\usepackage{listings} % code aus datei einbinden
\usepackage{scrhack}
\usepackage{comment}
% \usepackage{minted} % package for swift

% Swift language definition
\lstdefinelanguage{swift}
{
  morekeywords={
    func,if,then,else,for,in,while,do,switch,case,default,where,break,continue,fallthrough,return,
    typealias,struct,class,enum,protocol,var,func,let,get,set,willSet,didSet,inout,init,deinit,extension,
    subscript,prefix,operator,infix,postfix,precedence,associativity,left,right,none,convenience,dynamic,
    final,lazy,mutating,nonmutating,optional,override,required,static,unowned,safe,weak,internal,
    private,public,is,as,self,unsafe,dynamicType,true,false,nil,Type,Protocol,
  },
  morecomment=[l]{//}, % l is for line comment
  morecomment=[s]{/*}{*/}, % s is for start and end delimiter
  morestring=[b]" % defines that strings are enclosed in double quotes
}

\definecolor{keyword}{HTML}{BA2CA3}
\definecolor{string}{HTML}{D12F1B}
\definecolor{comment}{HTML}{008400}

\lstset{
  language=swift,
  basicstyle=\ttfamily,
  showstringspaces=false, % lets spaces in strings appear as real spaces
  columns=fixed,
  keepspaces=true,
  keywordstyle=\color{keyword},
  stringstyle=\color{string},
  commentstyle=\color{comment},
}

% word wrap with a red arrow
\lstset{
  basicstyle=\ttfamily,
  columns=fullflexible,
  frame=single,
  breaklines=true,
  postbreak=\mbox{\textcolor{red}{$\hookrightarrow$}\space},
}

\graphicspath{{./}{./images/}}

% #################################################################

\hyphenation{Cha-otn-gsch-werl}
\setlength\headheight{1.75cm}

\ihead{\small{Hochschule Hof}}
\chead{}
\ohead{\includegraphics[height=0.05\textheight]{fh_logo}}
\pagestyle{scrheadings}


\setcounter{secnumdepth}{5}
\setcounter{tocdepth}{5}
\renewcommand{\arraystretch}{1}

\parskip0.5\baselineskip plus 0.125\baselineskip minus 0.25\baselineskip
\parindent0em

%\automark[section]{chapter}

\titlehead{\begin{center}\includegraphics[width=5cm]{fh_logo}\end{center}}
 \title{
  Weiterentwicklung der iOS Stundenplan App der Hochschule Hof \\[1em]
  Dokumentation, Spezifikation, Konstruktion
}


\author{Christian G. Pfeiffer, Johannes Franz, Normen Krug, Marcel Hagmann,\\ Patrick Niepel, Carl Philipp Knoblauch}

%  Vorgelegt bei Prof. Dr.  Sven Rill

\date{22.01.2018} %##Abgabedatum einfügen


\begin{document}
\maketitle
\pagenumbering{roman}
\tableofcontents

%\listoftables

\newpage
\pagenumbering{arabic}

%hier die einzelnen Punkte einfügen
\chapter{Projektablauf}
Johannes Franz \& Christian Pfeiffer

\section{Ausgangssituation}
%%todo: Beschreiben, dass die iOS App schon von unseren Vorgängern entwickelt worden ist und welche Funktionen diese (V1,V2,V3) hattte

Im Wintersemester 2016/2017 begann die Entwicklung der iOS Stundenplanapp mit Studierenden des Studiengangs Mobile Computing im 5. Semester im Rahmen des Moduls "Fortgeschrittene Themen der Swift 3 Programmierung". Am Ende dieses Semesters wurde die Version V1 im AppStore veröffentlicht. Im Sommersemester 2017 wurde die App weiterentwickelt und schlussendlich die Version 2.0 veröffentlicht, welche eine Background Fetch Funktionalität einführte, lokale Notifications bei Vorlesungsverlegungen und Verbesserungen bei der Erkennung von einzelnen Vorlesungsterminen einführte.

Im Wintersemester 2017/2018 übernahmen unsere Projektgruppe (Studierende des 5. Semesters) im Rahmen des Moduls "Fortgeschrittene Themen der Swift 3 Programmierung" die Pflege und Weiterentwicklung der iOS Stundenplanapp.


\newpage
\section{Überlegungen zu Projektbeginn}

\subsection{GitHub / Branches}
GitHub ist ein Plattform zur effektiven Versionsverwaltung von Softwareprojekten. Branches stellen gewisse Softwareversionen da. Das Verwenden mehrerer Branches ermöglicht es größeren Softwareteams gleichzeitig an verschiedenen Softwarefeatures zu arbeiten.
Für die iOS Stundenplanapp wurde ein passendes Branch Konzept vom Projekt Team ausgearbeitet.

Der Großteil der Kommunikation lief über Issues und den Project Tab von GitHub ab.

\subsubsection{Ausgangslage}

Zu Projektbeginn wurden folgende Branches vorgefunden:
\begin{itemize}
\item master
\item development (obsolete)
\item v2 (depricated)
\item v3 (aktuell auf Deployment Target 10.0 entspricht iOS Version 10)
\end{itemize}


Von den vorherigen Projektteams wurde in GitHub ein Wiki angelegt.
Link zum Wiki:\\
\url{https://github.com/HochschuleHofStundenplanapp/iOS-App/wiki}

Darin wurden in kurzer Form einige Auszüge aus den jeweiligen PDF Dokumenten zusammengetragen.




GitHub Branch Übersicht:\\
 \url{https://github.com/HochschuleHofStundenplanapp/iOS-App/branches}

\subsubsection{Verbesserungen}
Es wurde beschlossen, die vorherige Struktur beizubehalten. Dabei wurden Branches jeweils mit der Versionsnummer beschriftet.

Der "master" Branch wird immer mit der aktuellsten Version gefüllt.

Neue Branches sind eingeführt worden:
\begin{itemize}
\item v3.1 (Bugfixes, kleinere neue "Features". Der Branch bleibt auf Swift 3.1  / iOS 10.0)
\item v3.2 Weitere Bugfixes. Der Branch bleibt auf Swift 3.1  / iOS 10.0 %%##todo
\item v4 in Rahmen der Studienarbeit entwickelte Erweiterungen der App (Swift 4 / iOS 11)
\item v4-widget Subbranch der v4 zur Anpassung der App an ein Framework, welches Vorraussetzung für die Implementierung eines Widget war.
\end{itemize}

Es wurden auch Branches entfernt:
\begin{itemize}
\item development Der development Branch wurde als obsolet gekennzeichnet und deshalb entfernt.
\end{itemize}

Es wurde von der Projektgruppe entschieden beim iOS 10 Deployment Target zu bleiben, da zu Beginn des Semester iOS 11 erst veröffentlicht wurde und die Verteilung erst ein paar Monate dauert. Zudem wurde für einige ältere Geräte, wie dem iPhone 5, iPhone 5C, and iPad 4, der Support eingestellt.

Letztendlich wurde entschieden das Projekt auf Swift 4 zu aktualisieren, da dies unseren Nutzern eine bessere Performance brachte Es gab auch Effizienz Verbesserungen bei der String Manipulation.\\

Zu Projektbeginn entstand die Idee, Branches nach Features zu benennen. Dabei war Design, Siri, Kalendersynchronisation, etc. angedacht. In der Projektphase hat sich zwischenzeitlich bewährt, die Branches weiterhin nach Versionsnummern zu benennen und bei großen Änderungen der Versionsnummer ein Thema anzuhängen.


\section{Ziele für Version 4}
Folgende Aufgaben wurden als Ziel für die Version 4 festgelegt:
% Jede Funktion kurz Beschreiben, was es tut... Warum wir es brauchen Add Siri, weil ist ja dokumentiert - cpfeiffer
\begin{itemize}
\item Onboarding
\item Hausaufgaben Manager
\item Push Notifications
\item Widget
\item iOS 11 Design
\item Testkonzept für die App
\end{itemize}


\section{Teams}

Team 1: (Pfeiffer, Scheler)
\begin{itemize}
\item Design
\item Buxfixes
\item Swift 4 Konvertierung
\end{itemize}


Team 2: (Franz, Krug):
\begin{itemize}
\item Siri
\item Push Notifications
\item Anpassung der Schnittstelle
\end{itemize}


Team 3: (Hagmann, Knoblauch, Niepel):
\begin{itemize}
\item Verschiedene Fehlerbehebungen
\item Hausaufgaben Manager
\item Onboarding 
\item Universal App Color
\item Kalenderschnittstelle überarbeitet
\end{itemize}


Team 4: (Kusserow, Sonntag, Dümmlein):
\begin{itemize}
\item Stundenplan Verbesserungen
\item Erstellen eines Widgets
\end{itemize}


Team 5: (Pöhlmann):
\begin{itemize}
\item Testkonzept ausarbeiten und Testen nach Protokoll
\end{itemize}


\section{Kommunikation}
Zur Verständigung untereinander und Festlegung der einzelnen Aufgaben wurden verschiedene Kommunikationsarten und Plattformen verwendet.
\begin{itemize}
\item Chat Gruppe mit allen Beteiligten
\item Kommunikation während der Zeit in der Hochschule
\item GitHub (Issues, Project Sektion)
\item Scrum ähnliche Vorstellung neu eingebauter Funktionen zu Beginn jeder Vorlesung
\end{itemize}


\section{Projektfortschritt dokumentiert}
Mithilfe des Projektfeature in GitHub konnte immer der Überblick über Subtasks der einzelnen Teams behalten werden und so der Fortschritt der einzelnen Teams dokumentiert werden.

Projekt Tab in GitHub\\
\url{https://github.com/HochschuleHofStundenplanapp/iOS-App/projects}






\section{Fazit und Ausblick}
Zu Beginn des Semesters bemühte sich das Team den Nutzern der Version 3 weiter mit Service Updates und Fehlerbehebungen zu versorgen. Hierfür entstanden die Versionen V3.1 und V3.2, welche die Stabilität der Stundenplanapp verbesserten und akute Probleme wie das Fehlen des roten Beschreibungstextes bei Stundenplanänderungen behob.
Bis zu Ende des Semesters war die Projektgruppe in der Lage alle geplanten Features in die V4 zu implementieren. Darüber hinaus entwickelte ein Team mit der Erweiterung SwiftSoup einen Parser für die Hochschul-Homepage, welcher in der Lage ist die vorlesungsfreie Zeit zu auszulesen. Damit kann nun immer der richtige Stundenplan, für das entsprechende Halbjahr angezeigt werden.\\

Zum Ende der Projektphase entsprach die Version 4 der App einen auslieferbaren Zustand. Nur noch kleine Änderungen und die Übergabe der Server Komponente an den IT-Service zum Updaten der Produktiv Server Umgebung stehen einer Veröffentlichung im weg.
Wie auch bei der vorherigen Version werden zu Beginn des Sommersemester 2018 weitere Qualitätsverbesserungen und Fehlerbehebungen vorgenommen  werden müssen, damit die neue Version bei den Endnutzern als abgerundetes Produkt ankommt.
\chapter{Design}
Angelina Scheler \& Christian Pfeiffer

\section{Einleitung}
Eine weitere Aufgabe, die unsere Gruppe übernommen hat, war das Onboarding für die Stundenplan App. Im Onboarding kann der Nutzer gleich zum Start der App seine Einstellungen zu Fakultät, Studiengang, Semester, Vorlesung und Kalendersynchronisation vornehmen. Das Onboarding wird gestartet, wenn noch keine Angaben zu den genannten Einstellungen gemacht wurden.

\begin{itemize}
\item Neues Design
\item besseres Design
\item iOS Design
\end{itemize}

\begin{enumerate}
\item Fakultät (die die Farbe der App bestimmt)
\end{enumerate}


\section{Stundenplan}
Unsere Design Sachen sind sehr schön!
\subsection{Screen1}

\begin{figure}[ht]
	\centering
  \frame{ \includegraphics[width=0.3\textwidth]{Mein_Stundenplan} }
	\caption{Mein schönstes Mockup}
	\label{fig1}
\end{figure}

\section{Onboarding}
\section{Einstellungen}
\section{Widget}

\section{demo}
Eine weitere Aufgabe, die unsere Gruppe übernommen hat, war das Onboarding für die Stundenplan App. Im Onboarding kann der Nutzer gleich zum Start der App seine Einstellungen zu Fakultät, Studiengang, Semester, Vorlesung und Kalendersynchronisation vornehmen. Das Onboarding wird gestartet, wenn noch keine Angaben zu den genannten Einstellungen gemacht wurden.

\begin{itemize}
\item Neues Design
\item besseres Design
\item iOS Design
\end{itemize}

\begin{enumerate}
\item Fakultät (die die Farbe der App bestimmt)
\end{enumerate}

\begin{figure}[ht]
	\centering
  \frame{ \includegraphics[width=0.4\textwidth]{Mein_Stundenplan} }
	\caption{Mein schönstes Mockup}
	\label{fig2}
\end{figure}


\chapter{Erweiterung durch eine Siri Integration}
Johannes Franz \& Normen Krug

\section{Einleitung}
Da sprachbasierte Mensch-Maschinen-Interface immer beliebter und praktikabler werden, ist es naheliegend, dass die Stundenplan App um diese erweitert wird. Apple bietet mit Siri solch einen Sprachassistenten ein. Dieser ist tief im System eingebaut und wird daher von Apple gut unterstützt. 
Da es weiterhin das Ziel sein soll, die App zur Nutzung nicht öffnen zu müssen, ist eine Siri Integration der nächst logische Schritt zur Weiterentwicklung der Anwendung.

\begin{figure}[ht]
	\centering
  \frame{ \includegraphics[width=0.9\textwidth]{hype-cycle-for-emerging-technologies-2017} }
	\caption{hype cycle for technologies 2017}
	\label{hype}
\end{figure}


\section{Istzustand}
Die App bietet aktuell keine Unterstützung für Sprachbefehle an. Eine Eingabe durch Sprachbefehle durch die Apple Watch ist so ebenfalls nicht möglich. Im Hinblick auf barrierefreie Bedienung hat die App daher noch Verbesserungspotential.

\section{Beispielhafte Anfragen}
Um die gängigen Anfragen abzudecken, müssen diese in der App vorher festgelegt werden. Um das Anliegen des Benutzers verstehen zu können müssen unterschiedliche Fragestellungen die zum selben Ergebnis führen abgedeckt werden.


\noindent%
\begin{tabularx}{\textwidth}{|p{.25\textwidth}|X| }
\hline
\textbf{Frage} & \textbf{Reaktion}  \\ \hline 

Wann/Wo ist meine nächste Vorlesung? & Name der Vorlesung, Zeitspanne und Raumnummer vorlesen.
UI zeigt diese Information noch einmal an.\\ \hline

Habe ich heute noch eine Vorlesung? & Ja/Nein Antwort, Name der Vorlesung mit Uhrzeit. 
Wenn ja wird im UI etwas angezeigt.
\\ \hline

Fällt heute etwas aus? & Ja/Nein Antwort, Name der Vorlesung mit Uhrzeit mit Grund. 
Wenn ja wird im UI etwas angezeigt.\\ \hline
Welche Änderungen gibt es heute? & Aufzählung der Änderungen für heute.
Auflisten dieser Änderungen im UI\\ \hline
In welchem Raum ist Vorlesung X? (optional/tricky) & Name der Vorlesung und Raum wird genannt.
Informationen im UI werden angezeigt.
\\ \hline

\end{tabularx}

\section{Umsetzung}
Apple bietet für die Umsetzung des Sprachassistenten das SiriKit an. Mittels einer “Applications Extension” namens “Intents Extension” ist es möglich Hooks für eine Reaktion der Stundenplan App zu definieren. Für die Ausgabe im Lockscreen kann dabei ein angepasstes UI zur Verfügung gestellt werden, welches die Sprachausgabe ergänzt.
\newline


\begin{figure}[ht]
	\centering
  \frame{ \includegraphics[width=0.9\textwidth]{Siri_Flowchart} }
	\caption{Siri Flowchart}
	\label{flowchart}
\end{figure}


\section{Voraussetzung für SiriKit}

Stand: 10.10.2017

Um SiriKit verwenden zu können, müssen die Kernbereiche der App in ein Framework ausgelagert werden. 
Apple empfiehlt das bei allen Erweiterungen.  Dieser Schritt ist notwendig, weil der Benutzer eine Interaktion mit Siri starten kann auch wenn die App zurzeit nicht läuft.  



\section{Einschränkungen}

Apple gewährt keinen vollständigen Zugriff auf Siri. Die zu entwickelte App muss in eine der Folgenden Kategorien/Domains fallen:
%\newline
\begin{itemize}
\item VoIP Calling 
\item Messaging
\item Payments
\item Lists and Notes
\item Visual Codes
\item Photos
\item Workouts
\item Ride Booking
\item Car Commands
\item CarPlay
\item Restaurant Reservations
\end{itemize}

\newpage

Die Schlüsselwörter welche Siri voraussetzt, um zu erkennen dass der Benutzer mit der App interagieren will, hängen von der Domain ab.
Diese Schlüsselwörter müssen zwingend in der Anfrage des Benutzer enthalten sein. 

Beispiel: 

In der „Search Message“  Domain müssen die Wörter „Suche“ und „Nachrichten“ enthalten sein. Falls eines der beiden Wörter nicht in der Anfrage enthalten ist, erkennt Siri die Anfrage nicht.  \newline
Passender Blog-Post zu den Thema: \newline
\url{https://swifting.io/blog/2016/07/18/20-sirikit-can-you-outsmart-provided-intents/}

Mögliche Workarounds: \newline
Es ist theoretisch möglichen mit der „Lists and Notes“ Domain, die Funktion für die Stundenplan hinzubiegen. Da aber die bestimmten Schlüsselwörter enthalten sei müssen, wird aber kein natürlich sprachliche Interaktion möglich sein.
\url{https://developer.apple.com/documentation/sirikit}

\section{Fazit}
Die vielversprechenden Idee die App um einen Sprachassistenten zu erweitern, ist zum aktuellen Zeitpunkt nicht realisierbar aufgrund der Limitierunegn vom SiriKit. Das Projekt musste an dieser Stelle unterbrochen werden und das Team musste sich neu orientieren.\\
Da Apple den Sprachassistenten Siri stets erweitert, kann allerdings darauf gehofft werden, dass einer Umsetzung zu einem späteren Zeitpunkt keine Barrieren mehr im Wege stehen.
\chapter{Push Notifications}
Johannes Franz \& Normen Krug

\section{Einleitung}
Push Notifications sind ein fester Bestandteil moderner Apps. Nutzer erwarten es häufig bei Änderungen oder Neuigkeiten informiert zu werden. Deswegen soll die Stundenplan App um solche erweitert werden. Ziel soll es sein den Benutzer über Stundenplanänderungen aktiv zu informieren, um speziell auf kurzfristige Änderungen reagieren zu können. Ein Server überprüft dabei eine Stundenplan Datenbank auf Änderungen und sendet eine Push Notification an alle iOS Geräte, die sich für die jeweilige Vorlesung registriert haben. Ein Apache Webserver stellt dabei mit einer MySQL Datenbank und entsprechenden Cronjobs das Backend bereit.

GitHub repository: \url{https://github.com/HochschuleHofStundenplanapp/iOS-App/}


\section{Istzustand}
Die iOS Version stellt nur lokale Push Notifications bereit. Dabei wird bisher keine Serverkomponente benötigt.

\newpage

\section{Projektablauf}
Um den Projektfortschritt nachvollziehen zu können, wurde dieser tabellarisch in Verbindung mit dem aktuellen Meilenstein aufgelistet.


\noindent%
\begin{tabularx}{\textwidth}{|p{.25\textwidth}|X| }
\hline
\textbf{Meilenstein} & \textbf{Details}  \\ \hline 

Vorlesungsbeginn & Themenfindung \\ \hline

Siri & Einarbeitung in Siri. Erkennen erster Hürden. Abbruch des Themas und dokumentieren der Erkenntnisse \\ \hline

Neue Themenfindung & Verwerfen von der Siri Projektidee und neue Themenfindung. \\ \hline

Festlegung & Thema Push Notifications festgelegt und Beginn der Einarbeitung. \\ \hline

Erfolgreiche Tests & Push Notifications lassen sich per PHP Script an einen fest eingestellten Token schicken \\ \hline

Neue Push Variante & Eine Test iOS App kann per php Script mittels MAMP lokal eine Push Notification senden. \newline
Datenbank lokal in PhpMyAdmin angelegt. \newline
PHP Script schreibt bei Aufruf in die angelegte SQL Datenbank. \newline
Einarbeitung und Konvertierung der Dokumentation in Latex.
 \\ \hline
 Register Script & Die Test iOS App wurde so erweitert, dass ein JSON File per POST Nachricht übermittelt werden kann.\newline
Das PHP Script parst nun das ankommende JSON File und fügt per insert die geparsten Daten in die Datenbank ein.\newline 
Einarbeitung in bestehende Schnittstelle und Überlegungen wie das bestehende Backend erweitert werden muss. 
 \\ \hline
 
 \
 

Abwärtskompatibilität & PHP Scripte der bestehenden (Android) Schnittstelle angepasst und getestet. \newline
\\ \hline 
 
 
Testserver & Testserver wurde bereitgestellt. \newline
\\ \hline 
 
 
PN mit HTTP2 & Push Notifications mit HTTP2 implementiert.\newline
\\ \hline 

Feinschliff & Finale Anpassung der Implementation und Dokumentation.
\\ \hline  

\end{tabularx}

\newpage


\section{Erster Ansatz der Umsetzung}

\subsection{MAMP als Virtuelle Umgebung}

\subsubsection{MAMP}
Die Testumgebung MAMP (Akronym steht für: Mac, Apache, MySQL, PHP) virtualisiert die genannten Komponenten, um lokale Tests zu ermöglichen.

\subsection{Hilfstools}
Die MacOS App ``Easy APNs Provider`` ermöglichte ersten Gehversuche, um Push Nachrichten an ausgewählte Token zu senden.
Zu beachten is,t dass der ``Easy APNs Provider`` das veraltete Push Interface benutzt. 
\\
Quelle: Mac App Store: ''easy apns provider push
 notification service testing tool''

\section{Vorbereitung der Umsetzung}
Apple bietet für die Umsetzung von Push Nachrichten den Apple Push Notification service (APNs) an. Dabei handelt es sich um einen bei Apple gehosteten Dienst, der Push Notifications per API ermöglicht. Nur diesem Dienst ist es erlaubt, die Push Notification direkt an das iOS Gerät zu senden.


\subsection{Server Installation}
Zur Installation der Software sind ROOT Rechte notwendig. Der \textit{wget} Befehl läd dabei eine zum aktuellen Zeitpunkt sehr neue \textit{curl} Version herunter. Diese hat die Besonderheit HTTP2 zu unterstützen, welches für die Push Notification Schnittstelle von Apple vorausgesetzt ist.

\newpage

\subsection{MySQL Datenbank}

Die Tabelle \textit{fcm\_nutzer} enthält eine Zuordnung von abonnierten Vorlesungsverlegungen mit dem jeweiligen Token des Gerätes. Dabei ist zusätzlich vermerkt, welches Betriebssystem verwendet wird. Die \textit{0} in der Spalte \textit{os} steht dabei für Android während \textit{1} iOS repräsentiert. \\
Da die ausgewählte Sprache des Nutzers nicht bekannt ist, wird für die \textit{language} Spalte an dieser Stelle auch null akzeptiert. Um bei einer späteren Version der App zwischen Nutzern die noch keine Sprache ausgewählt haben und allen Anderen differenzieren zu können, wird hier trotz der aktuell nur in deutsch vorhandenen App der Wert nicht standardmäßig auf deutsch gesetzt.
\lstinputlisting[language=SQL]{content/pushNotifications_create.sql}
Da eine bestehende Infrastruktur die Grundlage dieses Projektes darstellt, musste diese Tabelle lediglich um \textit{os} und \textit{language} erweitert werden.\\
Auf Änderungen an anderen Tabellen konnte komplett verzichtet werden.

\newpage

\subsubsection{fcm\_nutzer Tabelle}
Diese Tabelle beinhaltet vor allem die Verknüpfung zwischen dem eingetragenen Token (\textit{token}), der abonnierten Vorlesung (\textit{vorlesung\_id}) und dem mobilen Betriebsystem (\textit{os}). Diese Tabelle wurde beim Testsystem und beim Produktivsystem bereits um die neue Spalte \textit{os} und \textit{language} erweitert. Für letztere wurde die Schnittstelle, iOS und Android Version angepasst und getestet. Diese Funktion kann für zukünftige Implementierungen verwendet werden.
\begin{figure}[H]
	\centering
  \frame{ \includegraphics[width=1.0\textwidth]{datenbank_nutzer} }
	\caption{Tabelle fcm\_nutzer in phpmyadmin}
	\label{datenbank_nutzer}
\end{figure}

\textcolor{black}{
\lstinputlisting[language=bash]{content/pushNotifications_server_install.bsh}
}

\newpage

\subsection{Zertifikate}
Um mit einer gesicherten Verbindung auf die Apple Push Notification Schnittstelle zuzugreifen, werden zwei Zertifikate benötigt. Dabei handelt es sich um ein öffentliches und privates Zertifikat. Diese Zertifikate müssen vor der Verwendung in das Zielformat \textit{.pem} umgewandelt werden, bevor sie benutzbar sind.

Um die Zertifikate anzulegen und zu verwalten, stellt Apple eine Übersicht dem bei Apple angemeldeten Entwickler bereit. Bei dem verwendeten Zertifikat wird zwischen \textit{Development} und \textit{Production} unterschieden.\\
\url{https://developer.apple.com/account/ios/certificate/}


\subsection{CURL}
Der Server verwendet CURL um die Push Notificiation an den Service von Apple zu senden.  
CURL verwendet dabei das HTTP2 Protokoll.

\textcolor{black}{
\lstinputlisting[language=bash]{content/pushNotifications_curl_example.bsh}
}


\subsection{Sicherheit}
Das Thema Sicherheit spielt in der IT eine Zentrale, aber oft vernachlässigte Rolle. So ist beim der Einrichtung auf dem Server auf einige Punkte zu achten. Die hier verwendeten Software Versionen wie Apache2, MySQL oder CURL können auf dem Server auf die in der Zukunft aktuelle Version geupdatet werden.
Die Dateirechte der PHP Dateien sowie Zertifikate müssen entsprechend angepasst werden und nur gewissen Usern angehören.\\
Da das Projekt öffentlich auf GitHub für jeden einsehbar ist, müssen die Passwörter so gewählt werden, dass sie keinem vorher verwendeten Passwort gleichen. Dies war bisher im Umgang mit dem Testsystem absichtlich nicht der Fall, was praktische Vorteile mit sich gebracht hat.

\newpage

\subsection{Builden der App}
Bevor die App verwendet werden kann, muss speziell im Fall der Push Notifications auf gewisse Details geachtet werden.


In der \textit{Info.plist} stehen drei neue Parameter für Push Notifications bereit. Diese ermöglichen das Hin- und Herschalten zwischen dem Test- und Produktivserver mittels des \textit{isPushTesting} Parameters. Die beiden anderen Parameter repräsentieren sprechend den jeweiligen Link.\\
Bei dieser Einstellung ist darauf zu achten, dass sie sich nur auf die Anmeldung am jeweiligen Server bezieht. Informationen wie Stundenpläne und Änderungen kommen unabhängig von dieser Einstellung weiterhin vom Produktivserver (\textit{ https://app.hof-university.de/soap/}).

\begin{figure}[H]
	\centering
  \frame{ \includegraphics[width=1.0\textwidth]{plist} }
	\caption{Anpassungen in der Info.plist}
	\label{plist}
\end{figure}


Das Developer Team muss wie folgt ausgewählt werden, damit Push Notifications zur Verfügung stehen:
\begin{figure}[H]
	\centering
  \frame{ \includegraphics[width=1.0\textwidth]{devteam} }
	\caption{Wahl des Developer Teams}
	\label{devteam}
\end{figure}


Capabilities müssen wie folgt angepasst werden:

\begin{figure}[H]
	\centering
  \frame{ \includegraphics[width=1.0\textwidth]{capabilities} }
	\caption{Angepasste Capabilities}
	\label{capabilities}
\end{figure}

\newpage

\section{Implementierung}
Da die App bisher keine Push Notifications aus externen Quellen verwendet hat, musste dies komplett neu implementiert werden. Dabei mussten Änderungen am Server und  an der App vorgenommen werden, die gut aufeinander abgestimmt werden mussten.

\subsection{Senden der Volesungsdaten zum Server}
Damit der User für alle seine Vorlesungen über Änderungen informiert werden kann, müssen alle Vorlesungen gebündelt mit ihrem Device Token zum Server gelangen. 

\subsubsection{Extrahieren der Vorlesungs IDs}
Um an die Splusnamen der Vorlesungen zukommen, werden alle Vorlesungen aus den Model geladen. Die Splusnamen werden in ein \textit{JSONArray} gepackt und zusammen mit den Parametern Devicetoken, OS und Sprache in ein \textit{JSONObject} umgewandelt.
\lstinputlisting[language=swift]{content/lectures.swift}

\newpage

\subsubsection{POST Anfrage}
Die Daten werden per \textit{HTTP POST Request} an den Hochschule Server übermittelt. Alle benötigten Metatdaten werden der URLRequest Klasse übergeben. Zusätzlich wird der JSONPayload in den Body der Anfrage geschrieben. 
\lstinputlisting[language=swift]{content/PostRequest.swift}

\subsection{Server-Schnittstelle}
Da sich die Registrierung am Server von der bestehenden Android Schnittstelle unterschieden hat, mussten am Server verschiedene PHP Scripte angepasst werden.

Eine wichtige Anforderung an diese Implementierung war die Abwärtskompatibilität an noch nicht geupdatete Versionen der Android Stundenplan App. Auch die unkomplizierte Weiterverwendung von bestehenden Prozessen und Scripten war von Bedeutung.

\newpage


\subsection{Registerung für Push Notifications}
Die App registriert sich, sobald eine Vorlesung ausgewählt und übernommen wird am Server. Dafür ist das Script \textit{fcm\_register\_user.php} zuständig. Es erkennt anhand eines PHP Parameters ( \textit{...php?os=1 }), dass es sich um einen Aufruf durch eine iOS Anwendung handelt und stellt die passenden Abzweigungen und Funktionen bereit, die sich von der Android Variante unterscheiden.

Die dafür notwendigen Änderungen an der Datenbank wurden bereits im gleichnamigen Kapitel beschrieben. Diese Änderungen sind bereits auf dem Test- und Produktivsystem umgesetzt worden.

Durch die genannte Abwärtskompatibilität und Aufrufe des Scriptes durch interessierte Benutzer ist die Gefahr gegeben, dass \textit{Null} Werte in die Datenbank geschrieben werden. Das Script wurde dementsprechend erweitert, um möglichst an keiner Stelle diesen Fehlerfall zu ermöglichen.

\subsection{Senden der Nachrichten durch den Server}
Auf dem Server läuft ein Cronjob, der in einer definierten Zeit die Verlegungen prüft. Wurde eine Stundenplanänderung neu hinzugefügt, die noch nicht in der Tabelle \textit{fcm\_verlegungen} aufgelistet war, so reagiert das Script \textit{fcm\_update\_and\_send.php} und schreibt die Vorlesung in die genannte Tabelle.\\
Die Methode \textit{sendNotification()} prüft nun vor dem Senden um welche Übertragungsart (Firebase / APNs) es sich handelt und ruft die passende Methode auf.
Um der Übersichtlichkeit beizutragen, ist die APNs Funktionalität in der eingebundenen PHP Datei \textit{apnsPushIOS.php} abgebildet. Die darin enthaltene \textit{sendIosPush(...)} Methode die hier aufgerufen wird, benötigt lediglich \textit{title, body} und \textit{token} als Parameter. Der \textit{title} stellt dabei die Überschrift dar, während der \textit{body} den Nachrichteninhalt bereitstellt.

\newpage

\subsection{Reagieren auf Benachrichtigungen in der App}
Damit der Benutzer nicht jede Benachrichtigung einzeln bestätigen muss, wurde die \textit{didReceiveRemoteNotification()} Methode implementiert. Diese Funktion wird vom System aufgerufen wenn eine Benachrichtigung eintrifft. Die App kann beim Eintreffen der Benachrichtigung in zwei von drei verschieden Zuständen sein. Falls sie aktuell geöffnet oder inaktiv ist, kann der Benutzer mit einem Klick auf die Benachrichtigung schnell zu den Änderungen gelangen. 
Dabei werden alle weiteren Benachrichtigungen für die App ausgeblendet.
\\

\lstinputlisting[language=swift]{content/didRecivePush.swift}

\newpage

\section{Aufgetretene Probleme}
Als Herausforderung zu sehende Punkte haben häufig sehr viel Zeit in Anspruch genommen oder den Projektfortschritt entschleunigt.


\begin{itemize}
\item Verzögerte Bereitstellung des Servers führte zu komplexen Reimplementierungen in schwer zu synchronisierenden getrennten VMs
\item Beschränkte Rechte auf dem Testserver die nach und nach erweitert werden mussten
\item Verschiedene Zertifikatarten sind sehr verwirrend
\item Bundle Identifier und App Capabilities mussten nach jedem Git Pull wieder angepasst werden
\item Komplexes Testen (Lokale Server, Erreichbarkeit im Labornetzwerk / Wi-Fi, unterschiedliche Datei- und Serverstände) 
\item Zu Beginn kein Zugriff auf die Git Schnittstellen Projekt
\item Ablaufende Zertifikate
\item Serverseitig unterschiedliche Übertragungsverfahren zwischen Android und iOS
\item Erschwertes Debugging der PHP Scripte
\item Die auf den Server vorhanden Versionen von Apache2 und CURL, hatten kein HTTP2 unterstützt  
\item Struktur der Datenbank und der PHP Scripte war durch die Android App vorgegeben
\item Veraltete APNs Variante war stark verbreitet, brachte allerdings viele Probleme mit sich und musste als Ansatz letztendlich verworfen werden
\end{itemize}

\newpage

\section{Fazit}
Unser Team hat feststellen müssen, dass zu einem umfassenden iOS Projekt mehr als nur der Interface Builder und die Sprache Swift gehört. So sind Punkte zu nennen die positiv aus dem Projekt mitgenommen wurden.
\begin{itemize}
\item Linuxkenntnisse und der Umgang mit einem Produktiv- und Testsystem
\item PHP und Debugging auf dem Server
\item Beachten von Abhängigkeiten wie der Abwärtskompatibilität der Software zu früheren und bestehenden Android Versionen
\end{itemize}

\section{Weitere Arbeiten}
In diesem Projektabschnitt konnten alle gesteckten Ziele erreicht werden. Da solch ein relativ junges Projekt noch viele Möglichkeiten beinhaltet Funktionen zu verbessern und neue Funktionen einzuführen werden hier mögliche Punkte zur Anregung aufgelistet.

\begin{itemize}
\item Die Auswertung des Kommentarfeldes welche, aktuell bei der Android und iOS App auf dem Client Gerät stattfinden, könnten auf dem Server zentral bearbeitet werden. Das würde die Komplexität der beiden Apps verringern und erleichtert die Wartung des Codes an zentraler Stelle.
\item Erweiterung der Push Notification Information um eine Priorität, Ablaufdatum, etc.
\item Erweiterung der App und Schnittstelle um andere Sprachen wie z.B. Englisch was in der Android App bereits angeboten wird
\item Aufkommende Issues und Ideen aus dem öffentlichen Git Projekt der iOS App und Schnittstelle

\end{itemize}


\chapter{Testen der Anwendung}
\section{Test des Widgets}
Philipp Dümlein \& Maximilian Sonntag \& Bastian Kusserow

\section{Testfälle}
Das Widget wurde über einen längeren Zeitraum getestet. Dabei wurden folgende Testfälle erarbeitet und überprüft:

\begin{itemize}
\item Aktuell keine Vorlesung - beide Vorlesungen in der nächsten Woche
\item Aktuell keine Vorlesung - nächste Vorlesung am darauffolgenden Tag
\item Aktuell keine Vorlesung - nächste Vorlesung in der nächsten Woche
\item Aktuell keine Vorlesung - nächste Vorlesung übermorgen
\item Aktuell eine Vorlesung - nächste Vorlesung nächste Woche
\item Aktuell eine Vorlesung - nächste Vorlesung am darauffolgenden Tag
\item Aktuell eine Vorlesung - nächste Vorlesung danach
\item Aktuell eine Vorlesung - nächste Vorlesung übermorgen
\end{itemize}



\chapter{Gruppe PMP}
Patrick Niepel \& Marcel Hagmann \& Carl Philipp Knoblauch

\section{Einleitung}
Unsere Gruppe besteht aus Patrick Niepel, Marcel Hagmann und Carl Philipp Knoblauch. In diesem Team haben wir neben viel Debugging ...
\begin{itemize}
\item Aufgaben Erweiterung
\item Überarbeitung der Kalenderschnittstelle
\item Universal App Colors Funktion
\item Onboarding
\end{itemize}
erarbeitet.

\section{Überblick}
Eine Übersicht über unsere Arbeit über das Semester für Fortgeschrittene Programmierung unter Swift 3.

%\begin{comment}
\noindent%
\begin{tabularx}{\textwidth}{|p{.25\textwidth}|X|X| }
\hline
\textbf{Datum} & \textbf{Aufgaben/Vorlesung} & \textbf{Was wir gemacht haben}  \\ \hline 

KW 40 \newline 02.10.17 - 08.10.17 & 
• Einleitung der Vorlesung \newline 
• GitHub Projekt vorgestellt \newline 
• In Gruppen aufgeteilt \newline 
• Erster Bug vorgestellt \newline 
• \textbf{Neue Aufgabe:} Ersten Bug finden \newline 
• \textbf{Neue Aufgabe:} Themensuche \newline 
&
• Unsere Gruppe: Marcel Hagmann, Patrick Niepel, Carl Philipp Knoblauch \newline
• Marcel Hagmann findet ersten Bug \newline
• Themensuche Ideen: Aufgaben Erweiterung, Widget, Machine Learning, …\newline
• Ausarbeitung der Idee Aufgaben Erweiterung (Vorstellungen, Aufbau, Mockup) \newline
\\ \hline

KW 41 \newline 09.10.17 - 15.10.17 
&
• \textbf{Neue Aufgabe:} Aufgaben Erweiterung \newline 
& 

• Marcel Hagmann stellt ersten Bug vor und behebt ihn \newline
• Vorstellung der Aufgaben Erweiterung \newline
• Programmierung der Aufgaben Erweiterung \newline
 \\ \hline
 
 
KW 42 \newline 16.10.17 - 22.10.17 
&
• Weiter an der Aufgaben Erweiterung arbeiten \newline
&
• \textbf{Push:} BugFix von Marcel (16.10.17) \newline
• \textbf{Push:} Aufgaben Erweiterung (20.10.17) \newline
\\ \hline
 
 
KW 43 \newline 23.10.17 - 29.10.17 
&
• \textbf{Neue Aufgabe:} Kalenderschnittstelle \newline
&
• Weiterer Bug entfernt, daySize (23.10.17) \newline
• Komplette Umstrukturierung der Kalenderschnittstelle
\\ \hline


KW 44 \newline 30.10.17 - 05.11.17 
&
• Weiter an der Kalenderschnittstelle arbeiten \newline
&
• \textbf{Push:} Kalenderschnittstelle (4.11.17)
\\ \hline


KW 45 \newline 06.11.17 - 12.11.17 
&
• \textbf{Neue Aufgabe:} \newline Überarbeitung des Design und der Icons für die Aufgaben Erweiterung 
& 
• Bug gefunden: Datumsberechnungsfehler (nahezu Endlosschleife) \newline
• \textbf{Push (Bug):} Datumsberechnungsfehler behoben (10.11.17)\\ \hline


KW 46 \newline 13.11.17 - 19.11.17 
&
• \textbf{Neue Aufgabe:} App Color Design 
&
• Einarbeitung in Latex \newline
• Überarbeitung des Designs der Aufgaben Erweiterung \newline
• \textbf{Push:} des neuen Aufgaben Designs (13.11.17) \newline
• App Color Design (Farbe für App in Einstellungen auswählbar) \newline
• \textbf{Push:} App Color Design (17.11.17) \newline
\\ \hline





\end{tabularx}

%\end{comment}
\newpage

\noindent%
\begin{tabularx}{\textwidth}{|p{.25\textwidth}|X|X| }
\hline
\textbf{Datum} & \textbf{Aufgaben/Vorlesung} & \textbf{Was wir gemacht haben}  \\ \hline 

KW 47 \newline 20.11.17 - 26.11.17 
&
• \textbf{Neue Aufgabe:} Onboarding \newline
&
• Einarbeitung und Planung des Onboardings \newline
• Programmierung des Onboardings \newline
• Vorstellung des Onboardings (24.11.17) \newline
• Kalenderschnittstellen Debugging (KalenderID wurde nicht persistent gespeichert)
\\ \hline


KW 48 \newline 27.11.17 - 03.12.17 
&
• Weiter am Onboarding arbeiten \newline
&
• Programmierung des Onboardings \newline
• Debugging des Onboardings \newline
\\ \hline


KW 49 \newline 04.12.17 - 10.12.17 
&
• \textbf{Neue Aufgabe:} Aufgaben auch im Kalender anzeigen (Termin) \newline
• \textbf{Bug:} Einstellungen Synchronisation: Man kann nichts drücken \newline
• \textbf{Bug:} Beim App schließen bricht Kalendersynchronisation ab (wird nur zum teil ausgeführt)
&
• Programmierung: Aufgaben auch im Kalender anzeigen (In den Notizen) \newline
• Debugging \newline
• Einarbeitung in Latex \newline
\\ \hline
  
  
KW 50 \newline 11.12.17 - 17.12.17 
&
• \textbf{Bug:} Beim Löschen einer Aufgabe wird diese nicht aus den Notizen im Kalender entfernt
&
• Einarbeitung in Latex \newline
• Dokumentation schreiben \newline
\\ \hline


KW 51 \newline 18.12.17 - 24.12.17 
&
• \textbf{Bug:} Farben im Onboarding nicht richtig übernommen
&
• \textbf{Push:} Bei Änderungen an Tasks, werden diese nun auch in den Kalender übernommen (18.12.17)\newline
• Onboarding Farben gefixt \newline
\\ \hline


KW 2 \newline 08.01.18 - 14.01.18 
&
 • \textbf{Neue Aufgabe:} Push-Notifications in Onboarding integrieren \newline
 • \textbf{Neue Aufgabe:} Roter Punkt bei jeder Vorlesung, für die eine Aufgabe offen ist.
 &
 • Aufgaben Erweiterung überarbeitet (Rote Punkte) \newline
 • Onboarbing erweitert (Push-Notifications)
  \\ \hline


KW 3 \newline 15.01.18 - 21.01.18 
&
 ... 
 &
  test
  \\ \hline


KW 4 \newline 22.01.18 - 28.01.18 
&
Abgabe
 &
  test
  \\ \hline


\end{tabularx}

\section{Erster Bug}

\begin{itemize}
\item Wintersemester Sommersemester segmented Control ist verbuggt.
\item Änderungen werden gelöscht.
\item Alle Vorlesungen mit Kommentar werden in den Kalender geschrieben, aber alle anderen sind im Kalender nicht vorhanden.
\end{itemize}

\chapter{Aufgaben Erweiterung}
Patrick Niepel \& Marcel Hagmann \& Carl Philipp Knoblauch

\section{Einleitung}
Unsere erste Aufgabe war die Erweiterung der Stundenlpan App um ein Aufgaben Feature. Mit dem Aufgaben Feature kann der Nutzer seine Aufgaben aus Vorlesungen in die App eintragen, die dann mit dem Kalender synchronisiert werden.

\newpage
\section{Planung und Mockup}

Zuallererst machten wir uns Gedanken darüber, welche Informationen der Nutzer beim Hinzufügen seiner Aufgaben in die App angeben muss und möchte. Anhand dieser Erkenntnisse, fertigten wir das Mockup an.


\begin{figure}[ht]
	\centering
  \frame{ \includegraphics[width=0.9\textwidth]{Mockup_ios_aufgaben} }
	\caption{Mockup unserer Aufgaben Erweiterung}
	\label{fig1}
\end{figure}

\newpage
\section{Funktionen}
* Sortierung nach Datum und Fach
* Hinweis der noch zu erledigenden Aufgaben mit einem Badge in der Tab-Bar
* Hinzufügen, löschen und bearbeiten einer Aufgabe
* Vorlesungen in der Wochenübersicht, denen eine Aufgabe zugeteilt wurde hervorheben
* Aufgabe mit dem Kalender synchronisieren
* Speichern der Aufgaben in UserData


\chapter{Kalenderschnittstelle}
Patrick Niepel \& Marcel Hagmann \& Carl Philipp Knoblauch

\section{Einleitung}
Da Code öfters über die Zeit an Übersichtlichkeit und Korrektheit verliert, wie es in der Kalenderschnittstelle der Fall war, muss dieser in regelmäßigen Abständen kontrolliert und überarbeitet werden.

\section{Überarbeitung}
Eigentlich wollte das Team nur die folgenden Klassen überarbeiten:
\begin{itemize}
\item CalendarController
\item CalendarIntervace
\end{itemize}

Allerdings waren die Änderungen des Teams so groß, dass folgende Klassen auch davon betroffen waren:
\begin{itemize}
\item DateExtension
\item NotificationNameExtension
\item SettingsController
\item SettingsTableViewController
\end{itemize}

Als erstes überarbeitete das Team den einfachen Teil, die Übersichtlichkeit des Codes. Da Code von oben nach unten gelesen wird und beim Lesen des alten Codes viel hin und her gesprungen werden musste, ordnete das Team zu allererst die Reihenfolge der einzelnen Methoden an.

Danach überprüfte das Team jede Methode auf Logikfehler und bemerkten dabei, dass das Vorgänger-Team an der ein oder anderen Stelle gepfuscht haben, wodurch sich dann Folgefehler durch das gesamte Programm zogen.
Einer dieser Fehler war Beispielsweise der, dass in gewissen Vorlesungsstunden der Vorlesungsbeginn in der Klasse \textbf{JsonLectures}, falsch berechnet wird und so anstelle von 2017 das Jahr 0017 ausgegeben wurde, was dazu führte, dass die Semestervorlesungsstunden-Berechnung nicht von 9 Millionen (dass ein paar Sekunden dauert) von 6 Milliarden berechnet wurde und somit dazu führte, dass die App für den Nutzer “einfriert”.
Das Team fand diesen Fehler und entfernte diesen und damit auch die dadurch unnötig gewordenen Notifications, die die Vorgänger dazu verwendeten um vorzeitig aus dieser “fast Endlosschleife” auszubrechen.

Des Weiteren sorgte das Team für eine bessere Behandlung der Fehler die während der Ausführung der Operationen auftreten konnten, dass die Kalendersynchronisation im Hintergrund auch nach dem Schließen der App weiter läuft, der Kalender anhand seiner ID gesichert wird und der alte Kalender dementsprechend durch den neuen ausgetauscht wird.

\chapter{Onboarding}
Patrick Niepel \& Marcel Hagmann \& Carl Philipp Knoblauch

\section{Einleitung}
Eine weitere Aufgabe, die diese Gruppe übernommen hat, war das Onboarding für die Stundenplan App. Im Onboarding kann der Nutzer gleich zum Start der App seine Einstellungen zu Fakultät, Studiengang, Semester, Vorlesung und Kalendersynchronisation vornehmen. Das Onboarding wird gestartet, wenn noch keine Angaben zu den genannten Einstellungen gemacht wurden.


\section{Umsetzung}
Zuerst überlegte das Team uns, welche Informationen die App vom Nutzer benötigt, damit der Dienst im vollen Umfang verwendet werden kann.
Dabei kam das Team auf folgendes Ergebnis:
\begin{enumerate}
\item Fakultät (die die Farbe der App bestimmt)
\item Abfrage der Fakultät
\item Abfrage des Studiengangs
\item Abfrage des Semesters
\item  Abfrage der besuchten Vorlesungen
\item Ob eine Synchronisation mit dem Kalender gewünscht ist
\item Ob Push-Benachrichigungen gewünscht sind
\end{enumerate}

Das gesamte Onboarding befindet sich in einem separaten Storyboard (Onboarding.storyboard), damit die Übersichtlichkeit des aktuellen Projektes weiterhin gewährleistet wird.
Jeder Schritt im Onboarding (Ausnahme: Kalendersynchronisation) kann nur fortgesetzt werden, wenn eine Auswahl getroffen worden ist, die für die weiteren Schritte zwingend notwendig ist.
\chapter{Universal App Colors}
Patrick Niepel \& Marcel Hagmann \& Carl Philipp Knoblauch

\section{Einleitung}
Bis her hatte die Stundenplan App für jeden Tab eine andere Farbe. Wir haben uns darum gekümmert, dass der Nutzer in den Einstellungen zwischen den Hochschulfarben auswählen kann und diese dann für die ganze App übernommen werden. Die Hochschulfarben repräsentieren die Fakultäten an der Hochschule. Hat der Nutzer sich im Onboarding für eine Fakultät entschieden, wird auch die Farbe der App an die Farbe der Fakultät angepasst.


%usw. ...


\listoffigures

\end{document}
