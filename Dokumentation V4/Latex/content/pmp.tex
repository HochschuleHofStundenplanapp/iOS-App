
\chapter{Gruppe PMP}
Patrick Niepel \& Marcel Hagmann \& Carl Philipp Knoblauch

\section{Einleitung}
Unsere Gruppe besteht aus Patrick Niepel, Marcel Hagmann, Carl Philipp Knoblauch. In diesem Team haben wir, neben viel Debugging die Erweiterung Aufgaben, die Überarbeitung der Kalenderschnittstelle, sowie die Universal App Colors Funktion und das Onboarding erarbeitet.

\section{Überblick}
Eine Übersicht über unsere Arbeit über das Semester für Fortgeschrittene Programmierung unter Swift 3.

%\begin{comment}
\noindent%
\begin{tabularx}{\textwidth}{|p{.25\textwidth}|X|X| }
\hline
\textbf{Datum} & \textbf{Aufgaben/Vorlesung} & \textbf{Was wir gemacht haben}  \\ \hline 

KW 40 \newline 02.10.17 - 08.10.17 & 
• Einleitung der Vorlesung \newline 
• GitHub Projekt vorgestellt \newline 
• In Gruppen aufgeteilt \newline 
• Erster Bug vorgestellt \newline 
• \textbf{Neue Aufgabe:} Ersten Bug finden \newline 
• \textbf{Neue Aufgabe:} Themensuche \newline 
&
• Unsere Gruppe: Marcel Hagmann, Patrick Niepel, Carl Philipp Knoblauch \newline
• Marcel Hagmann findet ersten Bug \newline
• Temensuche Ideen: Aufgaben Erweiterung, Widget, Machine Learning, …\newline
• Ausarbeitung der Idee Aufgaben Erweiterung (Vorstellungen, Aufbau, Mockup) \newline
\\ \hline

KW 41 \newline 09.10.17 - 15.10.17 
&
• \textbf{Neue Aufgabe:} Aufgaben Erweiterung \newline 
& 

• Marcel Hagmann stellt ersten Bug vor und behebt ihn \newline
• Vorstellung der Aufgaben Erweiterung \newline
• Programmierung der Aufgaben Erweiterung \newline
 \\ \hline
 
 
KW 42 \newline 16.10.17 - 22.10.17 
&
• Weiter an der Aufgaben Erweiterung arbeiten \newline
&
• \textbf{Push:} BugFix von Marcel (16.10.17) \newline
• \textbf{Push:} Aufgaben Erweiterung (20.10.17) \newline
\\ \hline
 
 
KW 43 \newline 23.10.17 - 29.10.17 
&
• \textbf{Neue Aufgabe:} Kalenderschnittstelle \newline
&
• Weiterer Bug entfernt, daySize (23.10.17) \newline
• Komplette Umstrukturierung der Kalenderschnittstelle
\\ \hline


KW 44 \newline 30.10.17 - 05.11.17 
&
• Weiter an der Kalenderschnittstelle arbeiten \newline
&
\textbf{Push:} Kalenderschnittstelle (4.11.17)
\\ \hline


KW 45 \newline 06.11.17 - 12.11.17 
&
• \textbf{Neue Aufgabe:} \newline Überarbeitung des Design und der Icons für die Aufgaben Erweiterung 
& 
• Bug gefunden: Datumsberechnungsfehler (nahezu Endlosschleife) \newline
• \textbf{Push (Bug):} Datumsberechnungsfehler behoben (10.11.17)\\ \hline


KW 46 \newline 13.11.17 - 19.11.17 
&
• \textbf{Neue Aufgabe:} App Color Design 
&
• Einarbeitung in Latex \newline
• Überarbeitung des Designs der Aufgaben Erweiterung \newline
• \textbf{Push:} des neuen Aufgaben Designs (13.11.17) \newline
• App Color Design (Farbe für App in Einstellungen auswählbar) \newline
• \textbf{Push:} App Color Design (17.11.17) \newline
\\ \hline





\end{tabularx}

%\end{comment}
\newpage

\noindent%
\begin{tabularx}{\textwidth}{|p{.25\textwidth}|X|X| }
\hline
\textbf{Datum} & \textbf{Aufgaben/Vorlesung} & \textbf{Was wir gemacht haben}  \\ \hline 

KW 47 \newline 20.11.17 - 26.11.17 
&
• \textbf{Neue Aufgabe:} Onboarding \newline
&
• Einarbeitung und Planung des Onboardings \newline
• Programmierung des Onboardings \newline
• Vorstellung des Onboardings (24.11.17) \newline
• Kalenderschnittstellen Debugging (KalenderID wurde nicht persistent gespeichert)
\\ \hline


KW 48 \newline 27.11.17 - 03.12.17 
&
• Weiter am Onboarding arbeiten \newline
&
• Programmierung des Onboardings \newline
• Debugging des Onboardings \newline
\\ \hline


KW 49 \newline 04.12.17 - 10.12.17 
&
• \textbf{Neue Aufgabe:} Aufgaben auch im Kalender anzeigen (Termin) \newline
• \textbf{Bug:} Einstellungen Synchronisation: Man kann nichts drücken \newline
• \textbf{Bug:} Beim App schließen bricht Kalendersynchronisation ab (wir nur zum teil ausgeführt)
&
• Programmierung: Aufgaben auch im Kalender anzeigen (In den Notizen) \newline
• Debugging \newline
\\ \hline
  
  
KW 50 \newline 11.12.17 - 17.12.17 
&
... 
&
test

\\ \hline


KW 51 \newline 18.12.17 - 24.12.17 
&
 ...
&
test 
\\ \hline


KW 52 \newline 25.12.17 - 31.12.17 
&
 ... 
&
 test
\\ \hline


KW 1 \newline 01.01.18 - 07.01.18 
&
 ... 
 &
  test
  \\ \hline


KW 2 \newline 08.01.18 - 14.01.18 
&
 ... 
 &
  test
  \\ \hline


KW 3 \newline 15.01.18 - 21.01.18 
&
 ... 
 &
  test
  \\ \hline


KW 4 \newline 22.01.18 - 28.01.18 
&
 ... 
 &
  test
  \\ \hline


\end{tabularx}

\section{Erster Bug}

* Wintersemester Sommersemester segmented Control ist verbuggt.
* Änderungen werden gelöscht.
* Alle Vorlesungen mit Kommentar werden in den Kalender geschrieben, aber alle anderen sind im Kalender nicht vorhanden.