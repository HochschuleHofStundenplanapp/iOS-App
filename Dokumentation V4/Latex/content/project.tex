\chapter{Projektablauf}
Johannes Franz \& Christian Pfeiffer

\section{Ausgangssituation}
%%todo: Beschreiben, dass die iOS App schon von unseren Vorgängern entwickelt worden ist und welche Funktionen diese (V1,V2,V3) hattte

Im Wintersemester 2016/2017 begann die Entwicklung der iOS Stundenplanapp mit Studierenden des Studiengangs Mobile Computing im 5. Semester im Rahmen des Moduls "Fortgeschrittene Themen der Swift 3 Programmierung". Am Ende des Semesters wurde die Version V1 in den AppStore veröffentlicht. Im Sommersemester 2017 wurde die App weiterentwickelt und schlussendlich wurde Version 2.0 veröffentlicht, welche eine Background Fetch Funktionalität einführte, lokale Notifications bei Vorlesungsverlegungen und Verbesserungen bei der Erkennung von einzelnen Vorlesungsterminen einführte.

Im Wintersemester 2017/2018 übernahmen die Studierenden des 5. Semesters im Rahmen des Moduls "Fortgeschrittene Themen der Swift 3 Programmierung" die Pflege und Weiterentwicklung der iOS Stundenplanapp.


\newpage
\section{Überlegungen zu Projektbeginn}

\subsection{GitHub / Branches}
GitHub ist ein Plattform zur effektiven Versionsverwaltung von Softwareprojekten. Branches stellen gewisse Softwareversionen da. Das Verwenden mehrerer Branches ermöglicht es größeren Softwareteams gleichzeitig an verschiedenen Softwarefeatures zu arbeiten.
Für die iOS Stundenplanapp wurde ein passendes Branch Konzept vom Projekt Team ausgearbeitet.

Der Großteil der Kommunikation lief über Issues und den Project Tab von Github ab.

\subsubsection{Ausgangslage}

Zu Projektbeginn wurden folgende Branches vorgefunden:
\begin{itemize}
\item master
\item development (obsolete)
\item v2 (depricated)
\item v3 (aktuell auf Deployment Target 10.0 entspricht iOS Version 10)
\end{itemize}


Von den vorherigen Projektteams wurde in GitHub ein Wiki angelegt.
Link zum Wiki:\\
\url{https://github.com/HochschuleHofStundenplanapp/iOS-App/wiki}

Darin wurden im Notizartig und in kurzer Form einige Auszüge aus den jeweiligen PDF Dokumenten zusammengetragen.




GitHub Branch Übersicht:\\
 \url{https://github.com/HochschuleHofStundenplanapp/iOS-App/branches}

\subsubsection{Verbesserungen}
Es wurde beschlossen, die vorherige Struktur beizubehalten. Dabei wurden Branches jeweils mit der Versionsnummer beschriftet.

Der "master" Branch wird immer mit der aktuellsten Version gefüllt.

Neue Branches:
\begin{itemize}
\item v3.1 (Bugfixes, kleinere neue "Features". Der Branch bleibt auf Swift 3.1  / iOS 10.0)
\item v3.2 Weitere Bugfixes. Der Branch bleibt auf Swift 3.1  / iOS 10.0 %%##todo
\item v4 in Rahmen der Studienarbeit entwickelte Erweiterungen der App (Swift 4 / iOS 11)
\end{itemize}




Es wurde sich dafür entschieden beim iOS 10 deployment Target zu bleiben, um möglichst viele unter den Studierende verbeitete, alte Geräte zu unterstützen.

Version 4 wurde im laufe des Projekts auf Swift 4 geupgradet, bevor weitere Funktionen eingebaut wurden.

Zu Projektbeginn entstand die Idee, Branches nach Features zu benennen. Dabei war Design, Siri, Kalendersynchronisation, etc. angedacht. In der Projektphase hat sich zwischenzeitlich bewährt, die Branches weiterhin nach Versionsnummern zu benennen und zwischenzeitlich bei großen Änderungen ein Thema anzuhängen.




\section{Ziele für Version 4}
Folgende Aufgaben wurden als Ziel für die Version 4 festgelegt:
% Jede Funktion kurz Beschreiben, was es tut... Warum wir es brauchen Add Siri, weil ist ja dokumentiert - cpfeiffer
\begin{itemize}
\item Onboarding
\item Hausaufgaben Manager
\item Push Notifications
\item Widget
\item iOS 11 Design
\item Testkonzept für die App
\end{itemize}


\section{Teams}

Team 1: (Pfeiffer, Scheler)
\begin{itemize}
\item Design
\item Buxfixes
\item Swift 4 Konvertierung
\end{itemize}


Team 2: (Franz, Krug):
\begin{itemize}
\item Siri
\item Push Notifications
\end{itemize}


Team 3: (Hagmann, Knoblauch, Niepel):
\begin{itemize}
\item Verschiedene Fehlerbehebungen
\item Hausaufgaben Manager
\item Onboarding 
\item Universal App Color
\item Kalenderschnittstelle überarbeitet
\end{itemize}


Team 4: (Kusserow, Sonntag, Dümmlein):
\begin{itemize}
\item Stundenplan Verbesserungen


\item Erstellen eines Widgets
\end{itemize}


Team 5: (Pöhlmann):
\begin{itemize}
\item Testkonzept ausarbeiten und Testen nach Protokoll
\end{itemize}


\section{Kommunikation}
\begin{itemize}
\item Chat Gruppe mit allen Beteiligten
\item Kommunikation während der ZEit in der Hochschule
\item GitHub (Issues, Project Sektion)
\item Scrum ähnliche Vorstellung neu eingebauter Funktionen zu Beginn jeder Vorlesung
\end{itemize}


\section{Fazit des Projektes}
Herausforderungen:
\begin{itemize}
\item Überblick bewahren
\item Git Projekt und Issues im Blick zu haben
\end{itemize}


\section{Projektfortschritt dokumentiert}
Projekt Tab in Github\\
\url{https://github.com/HochschuleHofStundenplanapp/iOS-App/projects}
Scrum ähnliches arbeiten




\section{Ausblick}
% ganz runter setzten.

