\chapter{Projektablauf}
Johannes Franz \& Christian Pfeiffer

\section{Ausgangssituation}
%%todo: Beschreiben, dass die iOS App schon von unseren Vorgängern entwickelt worden ist und welche Funktionen diese (V1,V2,V3) hattte

Im Wintersemester 2016/2017 begann die Entwicklung der iOS Stundenplanapp mit Studierenden des Studiengangs Mobile Computing im 5. Semester im Rahmen des Moduls "Fortgeschrittene Themen der Swift 3 Programmierung". Am Ende dieses Semesters wurde die Version V1 im AppStore veröffentlicht. Im Sommersemester 2017 wurde die App weiterentwickelt und schlussendlich die Version 2.0 veröffentlicht, welche eine Background Fetch Funktionalität einführte, lokale Notifications bei Vorlesungsverlegungen und Verbesserungen bei der Erkennung von einzelnen Vorlesungsterminen integrierte..

Im Wintersemester 2017/2018 übernahmen unser Studienjahrgang (Studierende des 5. Semesters) im Rahmen des Moduls "Fortgeschrittene Themen der Swift 3 Programmierung" die Pflege und Weiterentwicklung der iOS Stundenplanapp.


\newpage
\section{Überlegungen zu Projektbeginn}

\subsection{GitHub / Branches}
GitHub ist eine Plattform zur effektiven Versionsverwaltung von Softwareprojekten. Branches stellen gewisse Softwareversionen dar. Das Verwenden mehrerer Branches ermöglicht es größeren Softwareteams gleichzeitig an verschiedenen Softwarefeatures zu arbeiten.
Für die iOS Stundenplanapp wurde ein passendes Branch-Konzept vom Projektmanagment Team ausgearbeitet.

Der Großteil der Kommunikation lief über Issues und den Project Tab von GitHub ab.

\subsubsection{Ausgangslage}

Zu Projektbeginn waren folgende Branches vorhanden:
\begin{itemize}
\item master
\item development (obsolete)
\item v2 (depricated)
\item v3 (aktuell auf Deployment Target 10.0 entspricht iOS Version 10)
\end{itemize}


Von den vorherigen Projektteams wurde in GitHub ein Wiki angelegt.
Link zum Wiki:\\
\url{https://github.com/HochschuleHofStundenplanapp/iOS-App/wiki}

Darin wurden in kurzer Form einige Auszüge aus den jeweiligen PDF Dokumenten zusammengetragen.




GitHub Branch Übersicht:\\
 \url{https://github.com/HochschuleHofStundenplanapp/iOS-App/branches}

\subsubsection{Verbesserungen}
Es wurde beschlossen, die vorherige Struktur beizubehalten. Dabei wurden Branches jeweils mit der Versionsnummer beschriftet.

Der "master" Branch wird immer mit der aktuellsten Version gefüllt.

Neue Branches sind eingeführt worden:
\begin{itemize}
\item v3.1: (Bugfixes, kleinere neue "Features". Der Branch bleibt auf Swift 3.1  / iOS 10.0)
\item v3.2: Weitere Bugfixes. Der Branch bleibt auf Swift 3.1  / iOS 10.0 %%##todo
\item v4: in Rahmen der Studienarbeit entwickelte Erweiterungen der App (Swift 4 / iOS 11)
\item v4-widget: Subbranch der v4 zur Anpassung der App an ein Framework, welches Vorraussetzung für die Implementierung eines Widget war.
\end{itemize}

Veränderungen an bestehenden Branches:
\begin{itemize}
\item master: Der master Branch beeinhaltet immer die aktuell ausgelieferte Version aus dem Appstore.
\item development: Der development Branch wurde als obsolet gekennzeichnet und deshalb entfernt.
\end{itemize}


Es wurde von der Projektgruppe entschieden beim iOS 10 Deployment Target zu bleiben, da zu Beginn des Semester iOS 11 erst veröffentlicht wurde und die Verteilung erst ein paar Monate dauerte. Zudem wurde für einige ältere Geräte, wie dem iPhone 5, iPhone 5C, and iPad 4, der Support eingestellt, weshalb einige Studierende keine App Updates mehr erhalten würden.

Letztendlich wurde entschieden das Projekt auf Swift 4 zu aktualisieren. Dies brachte u.a. Effizienzverbesserungen bei der String Manipulation. Diese wird beispielsweise in der Artificial Intelligence Klasse verwendet.

Zu Projektbeginn entstand die Idee, Branches nach Features zu benennen. Dabei war Design, Siri, Kalendersynchronisation, etc. angedacht. In der Projektphase hat sich zwischenzeitlich bewährt, Branches weiterhin nach Versionsnummern zu benennen und bei großen Änderungen der Versionsnummer ein Thema anzuhängen.


\section{Ziele für Version 4}
Folgende Aufgaben wurden als Ziel für die Version 4 festgelegt:
% Jede Funktion kurz Beschreiben, was es tut... Warum wir es brauchen Add Siri, weil ist ja dokumentiert - cpfeiffer
\begin{itemize}
\item Migration des Projektes auf Swift 4
\item Onboarding
\item Hausaufgaben Manager
\item Push Notifications
\item Widget
\item iOS 11 Design
\item Testkonzept für die App
\item Auslagerung des Appmodels in ein Framework
\item Parsen von vorlesungsfreien Tagen von der HS Webseite
\end{itemize}


\section{Teams}

Team 1: (Pfeiffer, Scheler)
\begin{itemize}
\item Design
\item Buxfixes
\item Swift 4 Konvertierung
\item Universal App Color Persistenz
\item Design des Onboarding
\end{itemize}


Team 2: (Franz, Krug):
\begin{itemize}
\item Siri
\item Implementierung von Push Notifications
\item Erweiterung der Schnittstelle
\end{itemize}


Team 3: (Hagmann, Knoblauch, Niepel):
\begin{itemize}
\item Verschiedene Fehlerbehebungen
\item Hausaufgaben Manager
\item Onboarding 
\item Universal App Color
\item Kalenderschnittstelle überarbeitet
\end{itemize}


Team 4: (Kusserow, Sonntag, Dümmlein):
\begin{itemize}
\item Stundenplan Verbesserungen
\item Erstellen eines Widgets
\item Auslagerung des Appmodels in ein Framework
\item Parseen von vorlesungsfreien Tagen von der HS Webseite
\end{itemize}


Team 5: (Pöhlmann):
\begin{itemize}
\item Testkonzept ausarbeiten
\item Testen nach Protokoll
\end{itemize}


\section{Kommunikation}
Zur Verständigung untereinander und Festlegung der einzelnen Aufgaben wurden verschiedene Kommunikationsarten und Plattformen verwendet.
\begin{itemize}
\item Chat Gruppe mit allen Beteiligten
\item Kommunikation während der Zeit in der Hochschule
\item GitHub (Issues, Project Tab)
\item Scrum ähnliche Vorstellung neu eingebauter Funktionen zu Beginn jeder Vorlesung
\end{itemize}


\section{Projektfortschritt dokumentiert}
Unter Zuhilfenahme des Projektfeature in GitHub konnte immer der Überblick über Subtasks der einzelnen Teams behalten werden und so der Fortschritt der Gruppen dokumentiert werden.

Projekt Tab in GitHub\\
\url{https://github.com/HochschuleHofStundenplanapp/iOS-App/projects}






\section{Fazit und Ausblick}
Zu Beginn des Semesters wurden die Nutzer  mit Service Updates und Fehlerbehebungen versorgt. Hierfür entstanden die Versionen V3.1 und V3.2, welche die Stabilität der Stundenplanapp verbesserten und akute Probleme wie das Fehlen des roten Beschreibungstextes bei Stundenplanänderungen behob.

Bis zum Ende des Semesters waren wir in der Lage alle zielgesetzten Funktionalitäten in die V4 zu implementieren. Nur noch kleine Änderungen und die Übergabe der Server Komponente an den IT-Service zum Updaten der Produktiv Server Umgebung verzögerten die Veröffentlichung in den Appstore.
Wie auch bei der vorherigen Version werden zu Beginn des Sommersemester 2018 weitere Qualitätsverbesserungen und Fehlerbehebungen vorgenommen  werden müssen, damit die neue Version bei den Endnutzern als abgerundetes Produkt ankommt.