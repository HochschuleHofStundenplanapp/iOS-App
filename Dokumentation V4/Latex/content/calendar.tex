\chapter{Kalenderschnittstelle}
Patrick Niepel \& Marcel Hagmann \& Carl Philipp Knoblauch

\section{Einleitung}
Da Code öfters über die Zeit an Übersichtlichkeit und Korrektheit verliert, wie es in der Kalenderschnittstelle der Fall war, muss dieser in regelmäßigen Abständen kontrolliert und überarbeitet werden.

\section{Überarbeitung}
Eigentlich wollte das Team nur die folgenden Klassen überarbeiten:
\begin{itemize}
\item CalendarController
\item CalendarIntervace
\end{itemize}

Allerdings waren die Änderungen des Teams so groß, dass folgende Klassen auch davon betroffen waren:
\begin{itemize}
\item DateExtension
\item NotificationNameExtension
\item SettingsController
\item SettingsTableViewController
\end{itemize}

Als erstes überarbeitete das Team den einfachen Teil, die Übersichtlichkeit des Codes. Da Code von oben nach unten gelesen wird und beim Lesen des alten Codes viel hin und her gesprungen werden musste, ordnete das Team zu allererst die Reihenfolge der einzelnen Methoden an.

Danach überprüfte das Team jede Methode auf Logikfehler und bemerkten dabei, dass das Vorgänger-Team an der ein oder anderen Stelle gepfuscht haben, wodurch sich dann Folgefehler durch das gesamte Programm zogen.
Einer dieser Fehler war Beispielsweise der, dass in gewissen Vorlesungsstunden der Vorlesungsbeginn in der Klasse \textbf{JsonLectures}, falsch berechnet wird und so anstelle von 2017 das Jahr 0017 ausgegeben wurde, was dazu führte, dass die Semestervorlesungsstunden-Berechnung nicht von 9 Millionen (dass ein paar Sekunden dauert) von 6 Milliarden berechnet wurde und somit dazu führte, dass die App für den Nutzer “einfriert”.
Das Team fand diesen Fehler und entfernte diesen und damit auch die dadurch unnötig gewordenen Notifications, die die Vorgänger dazu verwendeten um vorzeitig aus dieser “fast Endlosschleife” auszubrechen.

Des Weiteren sorgte das Team für eine bessere Behandlung der Fehler die während der Ausführung der Operationen auftreten konnten, dass die Kalendersynchronisation im Hintergrund auch nach dem Schließen der App weiter läuft, der Kalender anhand seiner ID gesichert wird und der alte Kalender dementsprechend durch den neuen ausgetauscht wird.
