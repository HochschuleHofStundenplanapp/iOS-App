\chapter{Artificial Intelligence}
Die Artificial Intelligence enthält Methoden, die das Kommentar der jeweiligen Vorlesung überprüft. Falls es im Kommentar enthalten ist, erkennt sie ob die Vorlesung ein Blocktermin ist, alle zwei Wochen stattfindet, eine Aufzählung von Terminen enthält, nicht parsbar ist, wann die Vorlesung beginnt oder wann sie endet. Die Artificial Intelligence wird direkt aufgerufen, nachdem die Vorlesung in der Klasse JsonLecture vom Server abgefragt werden.

Im Folgenden werden die wichtigsten Methoden der Klasse aufgelistet.
\begin{itemize}
     \item iterationOfLecture(comment: String, start: Date, end: Date) -$>$ iterationState
     \item containsEnumeration(comment: String) -$>$ Bool
     \item checkPeriod(comment: String) -$>$ (String, String)
     \item checkStart(comment: String) -$>$ String
     \item checkEnd(comment: String) -$>$ String
     \item getNextCalendarWeekOrDate(comment: String, keyword: String, length: Int) -$>$ String     
     \item getPreviousCalendarWeekOrDate(comment: String, keyword: String, length: Int) -$>$ String
\end{itemize}

\section{iterationOfLecture(comment: String, start: Date, end: Date) -$>$ iterationState}
Gibt die verschiedenen Arten der Wiederholung einer Vorlesung zurück. Es gibt fünf verschiedene Rückgabewerte.
\begin{itemize}
     \item iterationState.notParsable  \\[0.5em]
     Bedeutet, dass das Kommentar Wörter enthält die das Programm nicht parsen kann.
     \item iterationState.calendarWeeks \\[0.5em]
     Bedeutet, dass das Kommentar eine Aufzählung von Terminen enthält.
     \item iterationState.individualDate \\[0.5em]
     Bedeutet, dass die Vorlesung ein Einzeltermin ist.
     \item iterationState.twoWeeks \\[0.5em]
     Bedeutet, dass die Vorlesung alle zwei Wochen stattfindet.
     \item iterationState.weekly \\[0.5em]
     Bedeutet, dass die Vorlesung jede Wochen stattfindet.
\end{itemize}

\section{containsEnumeration(comment: String) -$>$ Bool}
Falls das Kommentar eine Aufzählung von Terminen enthält gibt die Methode diese zurück.

\section{checkPeriod(comment: String) -$>$ (String, String)}
Durchsucht das Kommentar nach einer Zeitspanne. Ein Beispiel so einer Zeitspanne ist "KW 12 bis 26", dabei wird die erste Kalenderwoche als Starttermin und die zweite als Endtermin der Vorlesung genommen. Vorerst sucht es jedoch nach Zeitspannen die nichts mit den Vorlesungsterminen zu tun haben, beispielsweise "Online-Anmeldung 09.03. - 16.03.", und entfernt diese temporär aus dem Kommentar.

\section{checkStart(comment: String) -$>$ String}
Durchsucht das Kommentar nach einem neuen Startdatum der Vorlesung.

\section{checkEnd(comment: String) -$>$ String}
Durchsucht das Kommentar nach einem neuen Enddatum der Vorlesung.

\section{getNextCalendarWeekOrDate(comment: String, keyword: String, length: Int) -$>$ String}
Gibt das nächste Datum oder die nächste Kalenderwoche nach dem angegebenen Keyword zurück. Length gibt dabei die Länge an wie viel Zeichen die Methode nach dem Keyword suchen soll, bevor sie die Suche nach einem Datum oder einer Kalenderwoche abbricht.

\section{getPreviousCalendarWeekOrDate(comment: String, keyword: String, length: Int) -$>$ String}
Gibt das vorrausgehende Datum oder die vorrausgehende Kalenderwoche nach dem angegebenen Keyword zurück. Length gibt dabei an, welche Anzahl an Zeichen die Methode vor dem Keyword suchen soll, bevor sie die Suche nach einem Datum oder einer Kalenderwoche abbricht.